\documentclass[11pt]{syllabus}

\coursetitle{Seminar on Metropolitan Inequality}
\coursenumber{AS.230.612}
\author{Prof. Michael Bader}
\email{mbader@jhu.edu}
\office{Mergenthaler 540}


\term{\ }
\classtime{\ }
\classroom{\ }

\newcommand{\officehourtimes}{TBD}
\newcommand{\biblocation}{../bib/}

\usepackage[utf8x]{inputenc}
\usepackage[T1]{fontenc}

\begin{document}
\maketitle 

\section{Office Hours}
\officehourtimes

\section{Course Description}
This course considers the sociological forces that shape modern metropolitan inequality. We will investigate the social and spatial patterns of inequality and how sociologists analyze patterns of inequality. As part of this inquiry, we will consider how sociologists (and related disciplines) use different methods to investigate topics of study. We will also consider how multiple levels of social action, from individual decisions to global political-economic relationships, affect the lives of residents in metropolitan areas.  

I hope that the course serves as a point from which to \emph{start} the investigation of metropolitan inequality. A single course cannot provide all of the information you will need to become a scholar in the field. But, it's my goal to help you have a sense of the field so that you have a toehold from which you may continue your own investigation of the topic. 

The course has the following objectives: 

\begin{objectives}
\item describe contemporary patterns of metropolitan inequality in the United States;
\item summarize major historical patterns of change that have occurred in metropolitan areas;
\item evaluate claims to truth made with different forms of evidence; 
\item discern how empirical work contributes to sociological theory; and
\item build a foundation on which to build your own research
\end{objectives}

\section{Assignments \& Grading}
\subsection{Assignments}
\begin{description}
\item[Weekly Reading Memos.] You will share with the class a response memo every week. The response memo should have three parts.

The first part should summarize each of the readings. For each reading, I would like you to write the following:

\begin{itemize}
    \item A one or two sentence summary that describes the major argument and the data used to support the argument. Think of this as what you would write about the piece in a literature review on the topic.
    \item A sentence that identifies the previous work that the piece engages and what questions they take up from that previous work.
    \item A few sentences on the methods that the author uses to make the argument the piece advances
    \item One or two sentences that explain if, and why (or why not), you find the argument convincing
\end{itemize}

In the second part, I want you to synthesize across the readings for the week. This should be a paragraph where you think about how the pieces are in conversation with one another and with the previous literature as well as where the research points as future directions. 

In the final part, I want you to think about "warrants" for your own research (Katz 1997). These may be in the form of nuggets that the pieces mention that your work can help to clarify or the piece itself may point directly to issues you study in your work (and anything in between). You should determine the format of these. I often will pull a quote from the piece and then write a sentence or phrase describing how I see the quote as a warrant for what I am doing or plan to do.

\item[Semester Project.] I would like to meet with each of you individually to discuss a semester-long project that would be beneficial for you. By our second class meeting, I would like you to write a brief description of what you propose, how the project will help you in the pursuit of your research, and a rough plan regarding how you will finish the project. I will meet with you and we will come to an agreement on what the project will be and how you will deliver as a final project and as interim assignments. Part of the plan should include what feedback you would like to receive from me and from your peers, as we will periodically share with one another. You will give a 15-minute presentation on whatever you choose your project to be.

\end{description}

\noindent The assignments will be weighted as follows:

\begin{center}
\begin{tabular}{lr}
\textbf{Assignment} & \textbf{Weight}\\\toprule
Weekly memos & 50\% \\
Final project \& presentation & 50\% \\\bottomrule
\end{tabular}
\end{center}

\section{Schedule}

\week{Introduction \& Overview}

\week{Studying People and Place}
\begin{readings}
\item \bibentry{klinenberg_heat_2003}.
\item \bibentry{browning_neighborhood_2006}.
\item \bibentry{duneier_ethnography_2006}.
\item \bibentry{klinenberg_blaming_2006}.
\end{readings}

\subsection{Birth of the Modern American Metropolis}

\week{The Philadelphia Negro}
\begin{readings}
\item \bibentry{dubois_philadelphia_1996}.
\item \bibentry{anderson_sociology_2001}.
\item \bibentry{zuberi_dubois_2004}.
\item \bibentry{logan_philadelphia_2016}.
\end{readings}

\week{The City}
\begin{readings}
\item \bibentry{park_city_2019}. (please be sure to read intro by Rob Sampson)
\item \bibentry{frazier_negro_1957}.
\item \bibentry{heblich_eastside_2021}.
\item \bibentry{sampson_neighborhoods_1997}.
\end{readings}

\subsection{Racial Segregation}

\week{American Apartheid}
\begin{readings}
\item \bibentry{massey_american_1993}.
\item \bibentry{logan_global_2010}.
\item \bibentry{lichter_new_2015}.
\item \bibentry{bader_fragmented_2016}.
\end{readings}

\week{Cycle of Segregation}
\begin{readings}
\item \bibentry{krysan_cycle_2017}.
\item \bibentry{kye_detecting_2019}.
\item \bibentry{bruch_choice_2019}.
\item \bibentry{bader_integration_2021}.
\end{readings}

\subsection{Place and Poverty}

\week{Truly Disadvantaged}
\begin{readings}
\item \bibentry{wilson_truly_1987}.
\item \bibentry{small_presence_2006}.
\item \bibentry{aldrich_continuities_1976}.
\item \bibentry{baum-snow_did_2007}.
\end{readings}

\week{TBD}
% \item \bibentry{lacy_new_2016}.

\subsection{Educational Inequality}

\week{Common Ground}
\begin{readings}
\item \bibentry{lukas_common_1985}.
\end{readings}

\week{Inequality in the Promised Land}
\begin{readings}
\item \bibentry{lewis-mccoy_inequality_2014}.
\item \bibentry{owens_inequality_2016}.
\item \bibentry{figlio_suburbanization_2012}.
\item \bibentry{bischoff_school_2020}.
\end{readings}

\subsection{Central City Change}

\week{Black on the Block}
\begin{readings}
\item \bibentry{pattillo_black_2007}.
\item \bibentry{freeman_gentrification_2004}.
\item \bibentry{papachristos_more_2011}.
\item \bibentry{hwang_divergent_2014}.
\end{readings}

\week{Live and Let Live}
\begin{readings}
\item \bibentry{perry_live_2016}.
\item \bibentry{lumley-sapanski_planning_2017}.
\item \bibentry{abascal_love_2015}.
\item \bibentry{bader_shared_forth}.
\end{readings}

\week{Final Presentations}


\bibliographystyle{apalike}
\bibliographystyle{\biblocation asr}
\nobibliography{\biblocation metro-inequality}

\end{document}